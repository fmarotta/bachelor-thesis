\documentclass[../main.tex]{subfiles}
\begin{document}

\chapter{Introduction}
\labch{intro}

https://www.ebi.ac.uk/gwas/home

https://www.broadinstitute.org/news/after-decade-genome-wide-association-st

Genotype => Expression => Phenotype <= Environment

Actually, environment can influence gene expression and epigenome as 
well. Also the genome, for instance UV rays cause mutations. Most of the 
time the environmental effects are random, but not always (\eg UV rays, 
smoking...).

somewhere: Height and most other quantitative traits are influenced by 
many variants of small effects.

gamazon2015:Gwas have found many associations, but a large sample size 
is needed.

other limit of gwas: they study single variants, but sometimes the 
disease manifest only when there is a certain \textit{combination} of 
variants.

gamazon2015: Gwas on their own are not enough (cite 
https://www.nature.com/articles/nature08494). in particular, there is a 
missing link between the variant and the disease: how (not why, 
\textit{how}) does the variant make one individual more susceptible to a 
disease? It is not true that the nearest gene is always involved.

fine mapping? (it may be necessary also for TWAS.)

gamazon2015: Many SNPs are found in regulatory regions, as evinced by 
the fact that they overlap with DNaseI sites (is this true? read Gusev, 
A. et al. Regulatory variants explain much more heritability than coding 
variants across 11 common diseases. bioRxiv 004309 (21 April 2014).), 
and that they often are found in eQTL (see Nicolae, D.L. et al. 
Trait-associated SNPs are more likely to be eQTLs: annotation to enhance 
discovery from GWAS. PLoS Genet. 6, e1000888 (2010).)

gamazon2015: eQTL mapping has shown that intermediate phenotypes, 
especially gene expression, are important. New projects are generating 
huge amounts of epxression data: ENCODE, GTEx, GEUVADIS, to name a few.

gamazon2015: they provide a way to aggregate SNPs in a biologically 
meaningful way, since they combine those variants that influence the 
expression of a gene. In general, either the phenotype is influenced by 
a small number of variants (one in the extreme case of mendelian 
diseases) with a large effect, or by a large number of variants of small 
effect. PrediXscan should districate this entanglement.

marker1
marker2
			\} => gene expression => phenotype
marker3
marker4

gamazon2015: multiple testing problems reduced.

\section{Heritability}

http://www.cureffi.org/2013/02/04/how-to-calculate-heritability/

Many traits vary among the individuals in a population: height and hair 
colour are obvious ones, but also the number of fingers can be different 
in some pathological cases (see the amish communities in the USA). The 
heritability of a trait is the proportion of variance which can be 
explained with the genetic variance among individuals.

Genetic variance where? only at the loci associated to the trait? It 
should be so, otherwise we underestimate heritability.

There are two definitions of heritability:

\begin{itemize}
	\item \enquote{narrow sense heritability}, $h^2$ is the heritability 
		due to additive genetic factors.
	\item \enquote{broad sense heritability}, $H^2$ is the heritability 
		due to all genetic factors, taking into account dominance and 
		gene-gene interactions.
\end{itemize}

According to the additive model, one individual's having $m$ alleles (0, 
1 or 2) influences the phenotype of a factor $m a$, where $a$ is the 
  effect of the allele. For instance, if you have 1 copy of allele 
  \geno{a} then your height increases of 1 cm, and if you have 2 copies 
  it increases of 2 cm. In the additive model, each allele and genotype 
  is independent of the others.

There are many ways to estimate narrow-sense heritability. One is in 
selective breeding, where $R = h^2 S$: we start from a population with 
mean $\lambda$, select a subpopulation with mean $\mu$, then take the 
offspring of the subpopulation, which will have mean $\mu_1$. R is the 
difference between the mean of the offspring and the mean of the 
original population (\ie, $\mu_1 - \lambda$), that is a measure of the 
success of selection; S is the difference between the mean of the 
selected subpopulation and the mean of the original population (\ie $\mu 
- \lambda$), that is a measure of the selective pressure applied, if you 
want.

Another way to interpret heritability in the narrow sense is the 
following. Make a plot of parents heights vs. offspring heights. If 
there is perfect heritability, the height of a son is equal to the 
average heights of the two parents, so the plot will be a straight line 
y=x. In general, $h^2$ is the slope of the regression line.

One way to get rid of environmental effects is to compare monozygotic 
twins with dizygotic ones. MZ twins share the same environment and the 
same genotype, whereas DZ twins share the same environment but have 
different genotypes (albeit pretty similar).

Wikipedia

We assume that $P = G + E$, where P = phenotype, G = genetics, E = 
environment. Phenotypic variance can be expressed as follows: 

\begin{equation}
	Var(P) = Var(G) + Var(E) + 2 Cov(G,E)
\end{equation}

\begin{equation}
	H^2 = Var(G) / Var(P)
	h^2 = Var(A) / Var(P)
\end{equation}

Visscher 2006

Previous studies calculated genetic variance according to kinship (\ie 
siblings share 1/2 of the genome, cousins 1/8, and so on). Visscher, 
instead, relies on the actual genotype of the samples, as assessed with 
markers. They had some 3000 pairs of siblings with genotype information. 
First, they calculated IBD sharing for each pair, then they calculated 
the heritability of height.

https://www.ncbi.nlm.nih.gov/books/NBK22001/

A trait is heritable if the variation of the trait in the individuals of 
a population can be imputed to genes. Note that every gene plays a role 
in the development of a trait, but is the variation due to genes? For 
instance (by fmarotta), in a population of genetic clones there can be 
variability in a trait. In this case it is convenient to think about 
plants, for they are often propagated by vegetative methods, so each 
plant is genetically identical to the others; however, some plants may 
be better irrigated or manured, and hence grow taller. In this case, 
genes play a role in the "development" of height, but the variation in 
the trait is entirely due to environmental factors. The example made by 
the book I am following is this: `there is no environment in which cows 
will speak. But, although the particular language that is spoken by 
humans varies from nation to nation, that variation is totally 
nongenetic'.

In principle, if genes determine variations in phenotypes, then 
offspring should be more similar to their parents than to unrelated 
individuals. This can be expressed as a correlation between parents and 
offspring (or between siblings). In other words, if we have X = 
"phenotype of parent X" and Y(X) = "phenotype of offspring of parent X", 
then the plot of Y(X) should be a straight line with positive slope. 
This, however, is valid ONLY IF THE ENVIRONMENT IS NOT SHARED BETWEEN 
RELATIVES MORE THAN IT IS SHARED BETWEEN UNRELATED PEOPLE.

In order to estimate the heritability of a trait, we have to check 
whether individuals with different genetic markers have different 
phenotypes. If the phenotypes are different and the markers are 
different, then probably the markers are linked to genes that influence 
the phenotype (note that the markers are rarely involved directly in 
influencing the phenotype); on the other hand, if phenotypes differ but 
the markers are the same, then the trait is not heritable (otherwise it 
would have been inherited together with the markers).

There is also another sense in which heritability is not a measure of 
the role played by the genes during development: heritability is 
measured by taking into account all genetic variation, not variation in 
genes associated to the phenotype. This boggles me: we might see two 
individuals with different phenotypes and discover that they are 
genetically different, but maybe they differ at loci that have nothing 
to do with the phenotype! Perhaps, by using large cohorts, this 
phenomenon is less likely to appear, because different individuals will 
differ at different loci. It is also true that every locus influence 
every trait...

In experimental models, heritability is measured by artificial selection 
(see above).

https://www.nature.com/articles/ng0508-489

Combination of alleles may have specific effects both if they occur at 
the same locus (dominance) or at different loci (epistasis). 

There is a nice picture showing the statistical power of association 
studies as a function of effect size and population size.

One could find genes near markers associated to a phenotype and look for 
overrepresented pathways.

\enquote{The main conclusion emerging from the current studies is that 
GWAS are able to robustly identify common variants that are associated 
with height but that the effect sizes of individual variants are small, 
so that very large sample sizes are needed to detect associations 
reliably. Single laboratories are unlikely to have sufficient sample 
sizes to do powerful studies on their own, and the trend in human 
complex trait mapping has been to create consortia of research groups 
and even consortia of consortia.}

At the same time, among the full sequences now available there are so 
many variants that trying to associate them with anything is very 
difficult. Statistics is not enough in this case.

https://www.nature.com/articles/nrg2322

Heritability deals with the old nature-nurture debate, in particular 
with how offspring resemble parents.

P = G + E (taking account of sex, age and other covariates while 
defining varP)

varP = varG + varE (assuming there is no genotype by environment 
covariance. there would be covariance if intelligent parents would 
provide an intelligence-stimulating environment for their offspring, or 
if cattle would be fed according to production. Also, the interaction 
between genotype and environment is neglected, i.e. when the effect of 
the genotype depends on the environment. they are ignored because they 
are difficult to evaluate.)

H2 = var(G) / var(P)

varG = varA + varD + varE (additive, dominant, epistatic effects. 
covariates are assumed 0)

h2 = varA / varP

I have not understood this part:

\enquote{in a non-inbred population, half of the additive genetic 
variance is between families and half is within families. This implies 
that for a trait such as adult height in human populations, with a 
heritability of 0.8 and a standard deviation of approximately 7 cm in 
the population, the standard deviation of height in adult offspring 
around the mean value of the parents is 5.4 cm ($=sqrt[7^2 (1 – 
½*0.8)]$), which is not much smaller than the standard deviation in the 
entire population. Hence, tall parents have on average tall children, 
but with a considerable variation around the parental mean.}

Wait a minute: $h^2$ because the variance is a squared thing! $h$ would 
correspond to the standard deviation.

\enquote{Because individuals transmit only one copy of each gene to 
their offspring, most relatives share only single or no copies that are 
identical by descent (IBD) (the most important exceptions are identical 
twins and full siblings (sibs)), and dominance and other non-additive 
genetic effects that are based on sharing two copies do not contribute 
to their phenotypic resemblance. This is why the selection response and 
correlation of most relatives depend on h2 and not H2, and why h2 is the 
usual parameter.} That means that most people are heterozygous.

Breeder's equation: R = h2 S.

\end{document}
