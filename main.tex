% TUFTE-BOOK CLASS {{{

%%%%%%%%%%%%%%%%%%%%%%%%%%%%%%%%%%%%%%%%%
% Tufte-Style Book (Documentation Template)
% LaTeX Template
% Version 1.0 (5/1/13)
%
% This template has been downloaded from:
% http://www.LaTeXTemplates.com
%
% Original author:
% The Tufte-LaTeX Developers (tufte-latex.googlecode.com)
%
% Edited by:
% Federico Marotta (federicomarotta@mail.com)
%
% License:
% Apache License (Version 2.0)
%
% IMPORTANT NOTE:
% In addition to running BibTeX to compile the reference list from the .bib
% file, you will need to run MakeIndex to compile the index at the end of the
% document.
%
%%%%%%%%%%%%%%%%%%%%%%%%%%%%%%%%%%%%%%%%%

% This document uses the tufte-book class (which in turn uses the 
% tufte-common class)
%
% Summary of available options for the tufte-book class:
% a4paper: use the A4 paper size
% sfsitednotes: use a sans serif font for sidenotes and title blocks
% twoside: print the page number on the outside edge
% symmetric: typeset the sidenotes on the outside edge instead of on the right side
% justified: justify text
% notitlepage: have \maketitle generate a title block instead of a title page
% notoc: suppress tufte's custom table of contents
% nohyper: do not load the hyperref package
% nobib: do not load natbib (which is incompatible with biblatex)
\documentclass[a4paper,twoside,notoc,marginals=justified,nobib]{tufte-book} 

% }}}

% PACKAGES AND OTHER DOCUMENT CONFIGURATIONS {{{
% --------------------------------------------------------------------------

% Encoding and fonts
\usepackage[utf8]{inputenc} % Use utf8 encoding in the input (.tex) file
\usepackage[english]{babel} % Load characters and hyphenation
\usepackage[T1]{fontenc} % Use utf8 encoding in the output (.pdf) file
\usepackage{hyphenat} % Enable hyphenation for special fonts
\usepackage[english=british]{csquotes} % Quote like a boss

% Special contents
\usepackage{textgreek} % Greek characters
\usepackage{amsmath} % Improved mathematics
\usepackage{amsfonts} % Math fonts
\usepackage[scale=2]{ccicons} % Creative Commons icons
\usepackage{units} % Used for printing standard units
\usepackage{xspace} % Used for printing a trailing space better than using a tilde (~) using the \xspace command
\usepackage{todonotes} % Add useful notes in the margins
\usepackage{subfiles} % Adopt a modular structure
\usepackage[most]{tcolorbox} % Make fancy boxes

% TOC
\setcounter{secnumdepth}{-1} % Do not number headings
\setcounter{tocdepth}{1} % Show headings up to sections in the toc
%\usepackage[notlot,notlof]{tocbibind}
%\makeatletter
%\renewcommand*\l@chapter{\@dottedtocline{1}{0em}{2.3em}}
%\renewcommand*\l@section{\@dottedtocline{1}{0em}{2.3em}}
%\makeatother

% Images
\usepackage{graphicx} % Needed to insert images into the document
\setkeys{Gin}{width=\linewidth,totalheight=\textheight,keepaspectratio} % Improves figure scaling
\graphicspath{{images/}{../images/}} % Sets the default location of pictures
% (the second path is relative to the subfiles)

% Bibliography
% See https://github.com/Tufte-LaTeX/tufte-latex/issues/60 for the 
% problems concerning biblatex
\usepackage[
	citestyle=authortitle-icomp,
	bibstyle=apa,
	sorting=none,
	backend=biber,
	autocite=footnote,
	hyperref=false,
]{biblatex}
\addbibresource{main.bib}
\AtEveryBibitem{
	\clearfield{url} % Do not show URL
	\clearfield{issn}
	\clearfield{archivePrefix}
	\clearfield{arxivid}
}
%\renewcommand{\cite}[2][0pt]{\sidenote[][#1]{\fullcite{#2}} }

% Index
%\usepackage{makeidx} % Used to generate the index
%\makeindex % Generate the index which is printed at the end of the 
%document (remember to uncomment \printindex if you use the index)

% Document appearance
\usepackage{microtype} % Improves character and word spacing
\usepackage{fancyvrb} % Allows customization of verbatim environments
\usepackage{booktabs} % Better horizontal rules in tables
\usepackage{multicol} % Customisation for tables
\usepackage{caption} % Hyperlinks point to the top, not to the caption
\usepackage{varioref} % Reference also the page if necessary
\hypersetup{colorlinks} % Comment this if you don't want colored links
\hypersetup{linkcolor=black}
\fvset{fontsize=\normalsize} % Change here the font size of all verbatim 
\setlength{\headheight}{24pt} % Set the height of headers

% Print the chapter number and title on the same line 
% https://latex.org/forum/viewtopic.php?t=25750
\titleformat{\chapter}%
[block]% shape
{\relax\ifthenelse{\NOT\boolean{@tufte@symmetric}}{\begin{fullwidth}}{}}% format applied to label+text
{\itshape\huge\thechapter\hspace{10pt}}% label
{0pt}% horizontal separation between label and title body
{\huge\rmfamily\itshape}% before the title body
[\ifthenelse{\NOT\boolean{@tufte@symmetric}}{\end{fullwidth}}{}]% after the title body

% CUSTOM COMMANDS AND ENVIRONMENTS {{{
% --------------------------------------------------------------------------

% Document-specific {{{

% Print latin words in italics
\newcommand{\invitro}{\textit{in vitro}\xspace}
\newcommand{\invivo}{\textit{in vivo}\xspace}
\newcommand{\cis}{\textit{cis}\xspace}
\newcommand{\trans}{\textit{trans}\xspace}
\newcommand{\etal}{\textit{et al.}\xspace}
\newcommand{\denovo}{\textit{de novo}\xspace}
\newcommand{\etcetera}{\textit{et cetera}\xspace}
\newcommand{\etc}{\textit{etc.}\xspace}

% Easily label and reference elements
% Note that \label must appear after \caption
\newcommand{\labch}[1]{\label{ch:{#1} }}
\newcommand{\refch}[1]{Chapter \textit{\nameref{ch:{#1} }}\hairsp}
\newcommand{\labsec}[1]{\label{sec:{#1} }}
\newcommand{\refsec}[1]{Section \textit{\nameref{sec:{#1} }}\hairsp}
\newcommand{\labfig}[1]{\label{fig:{#1} }}
\newcommand{\reffig}[1]{Figure \vref{fig:{#1} }}
\newcommand{\labtab}[1]{\label{tab:{#1} }}
\newcommand{\reftab}[1]{Table \vref{tab:{#1} }}

% Environment to put external abstracts in a box
\newenvironment{external_abstract}[1]
{
	\begin{fullwidth}
	\begin{tcolorbox}[colback=white,colframe=gray!15,coltitle=black!90,arc=0pt,outer 
arc=0pt,#1]
	\footnotesize
}
{
	\end{tcolorbox}
	\end{fullwidth}
}

% }}}

\newcommand{\hlred}[1]{\textcolor{Maroon}{#1}} % Print text in maroon

\newcommand{\hangp}[1]{\makebox[0pt][r]{(}#1\makebox[0pt][l]{)}} % New command to create parentheses around text in tables which take up no horizontal space - this improves column spacing
\newcommand{\hangstar}{\makebox[0pt][l]{*}} % New command to create asterisks in tables which take up no horizontal space - this improves column spacing

\newcommand{\monthyear}{\ifcase\month\or January\or February\or March\or 
April\or May\or June\or July\or August\or September\or October\or 
November\or December\fi\space\number\year} % A command to print the current month and year

\newcommand{\openepigraph}[2]{ % This block sets up a command for printing an epigraph with 2 arguments - the quote and the author
\begin{fullwidth}
\sffamily\large
\begin{doublespace}
\noindent\allcaps{#1}\\ % The quote
\noindent\allcaps{#2} % The author
\end{doublespace}
\end{fullwidth}
}

\newcommand{\blankpage}{\newpage\hbox{}\thispagestyle{empty}\newpage} % Command to insert a blank page

\newcommand{\hangleft}[1]{\makebox[0pt][r]{#1}} % Used for printing commands in the index, moves the slash left so the command name aligns with the rest of the text in the index 
\newcommand{\hairsp}{\hspace{1pt}} % Command to print a very short space
\newcommand{\ie}{\textit{i.\hairsp{}e.}\xspace} % Command to print i.e.
\newcommand{\eg}{\textit{e.\hairsp{}g.}\xspace} % Command to print e.g.
\newcommand{\na}{\quad--} % Used in tables for N/A cells
\newcommand{\measure}[3]{#1/#2$\times$\unit[#3]{pc}} % Typesets the font size, leading, and measure in the form of: 10/12x26 pc.
\newcommand{\tuftebs}{\symbol{'134}} % Command to print a backslash in tt type in OT1/T1

% }}}

% }}}

% BOOK META-INFORMATION {{{
%---------------------------------------------------------------------------

% https://tex.stackexchange.com/questions/179839/tufte-book-title-page-customization

\newsavebox{\logo}
\savebox{\logo}{\includegraphics[height=3cm]{logo_unito.png}}

% Title of the book
\title[Bridging the Gap between Genome, Transcriptome and 
Disease]{Transcriptome-Wide Association Studies: Bridging the Gap 
	between Genome, Transcriptome and Disease} 

\author[FM]
{
	Supervisor: Prof. Paolo Provero
	\hfill
	Candidate: Federico Marotta
}

\publisher
{
	\usebox{\logo} \par \vspace{1cm}
	\smallcaps{Università degli Studi di Torino} \par
	\smallcaps{Scuola di Medicina} \par
	\smallcaps{Dipartimento di Biotecnologie Molecolari e Scienze per la Salute} \par
	\smallcaps{Corso di Laurea in Biotecnologie}
} 

\date{Tesi di Laurea, 18 Luglio 2018}

% }}}

% FRONT MATTER {{{
%---------------------------------------------------------------------------

\begin{document}

\frontmatter

% EPIGRAPH
%---------------------------------------------------------------------------

%\thispagestyle{empty}
%\openepigraph{The public is more familiar with bad design than good 
%design. It is, in effect, conditioned to prefer bad design, because 
%that is what it lives with. The new becomes threatening, the old 
%reassuring.}{Paul Rand, {\itshape Design, Form, and Chaos}}
%\vfill
%\openepigraph{A designer knows that he has achieved perfection not when 
%there is nothing left to add, but when there is nothing left to take 
%away.}{Antoine de Saint-Exup\'{e}ry}
%\vfill
%\openepigraph{\ldots the designer of a new system must not only be the 
%implementor and the first large-scale user; the designer should also 
%write the first user manual\ldots If I had not participated fully in 
%all these activities, literally hundreds of improvements would never 
%have been made, because I would never have thought of them or perceived 
%why they were important.}{Donald E. Knuth}

% TITLE PAGE
%---------------------------------------------------------------------------

% 1. the default way
%\maketitle % Print the title page

% 2. The custom title page
\makeatletter
\renewcommand{\maketitlepage}{%
\begingroup%
\begin{fullwidth}
\begin{centering}
\setlength{\parindent}{0pt}

\fontsize{13}{13}\selectfont\textit{\@publisher}\par

\null\vspace{\stretch{1}}

{\fontsize{22}{40}\selectfont\@title\par}

\vspace{\stretch{1}}

{\fontsize{16}{16}\selectfont\textit{\@author}\par}

\vspace{\stretch{1}}\null

{\fontsize{13}{13}\selectfont\textsf{\smallcaps{\@date}}\par}

\thispagestyle{empty}
\end{centering}
\end{fullwidth}
\endgroup
}
\makeatother
\maketitle % Print the title page

% COPYRIGHT PAGE
%---------------------------------------------------------------------------

\newpage
%\begin{fullwidth}
~\vfill
\thispagestyle{empty}
\setlength{\parindent}{0pt}
\setlength{\parskip}{\baselineskip}

\ccbysa

This work is licensed under a 
\href{http://creativecommons.org/licenses/by-sa/4.0/} {Creative Commons 
	Attribution-ShareAlike 4.0 International License}, and typeset with 
\href{ttps://www.latex-project.org/}{\LaTeX} using the 
\href{https://tufte-latex.github.io/tufte-latex/}{Tufte-LaTeX} class.

%\par\smallcaps{Published by \thanklesspublisher}
%\par\smallcaps{tufte-latex.googlecode.com}
%\par Licensed under the Apache License, Version 2.0 (the ``License''); 
%you may not use this file except in compliance with the License. You 
%may obtain a copy of the License at 
%\url{http://www.apache.org/licenses/LICENSE-2.0}. Unless required by 
%applicable law or agreed to in writing, software distributed under the 
%License is distributed on an \smallcaps{``AS IS'' BASIS, WITHOUT 
%WARRANTIES OR CONDITIONS OF ANY KIND}, either express or implied. See 
%the License for the specific language governing permissions and 
%limitations under the License.\index{license}
%\par\textit{First printing, \monthyear}
%\end{fullwidth}

% TABLES OF CONTENTS, FIGURES, TABLES
%---------------------------------------------------------------------------

% Decrease line spacing so that the toc fits in one page.
% https://tex.stackexchange.com/questions/56546/how-to-change-spaces-between-items-in-table-of-contents
\renewcommand{\baselinestretch}{0.75}\normalsize
\tableofcontents % Print the table of contents
%\listoffigures % Print a list of figures
%\addcontentsline{toc}{chapter}{List of Figures}
%\listoftables % Print a list of tables
%\addcontentsline{toc}{chapter}{List of Tables}
%\listoftodos
\renewcommand{\baselinestretch}{1.0}\normalsize

% DEDICATION PAGE
%---------------------------------------------------------------------------

\cleardoublepage
\null\vspace{\stretch{1}}
\begin{doublespace}
\noindent\fontsize{15}{15}\selectfont\itshape
\nohyphenation

%\enquote{The harmony of the world is made manifest in Form and Number, 
%and the heart and soul and all the poetry of Natural Philosophy are 
%embodied in the concept of mathematical beauty.}\\
%---D'Arcy Wentworth Thompson, \textnormal{On Growth and Form}

\vspace{\stretch{1}}

\noindent\fontsize{15}{15}\selectfont\itshape

I would like to thank my supervisor, Prof. Paolo Provero, and all the 
former and current members of the Computational Biology Unit whom I have 
met: Elisa Mariella, Davide Marnetto, Elena Grassi, Ugo Ala, Alessandro 
Lussana, Stefano Gilotto. Their help has often been invaluable and they 
have never failed to provide a stimulating environment for me to work 
in.

\end{doublespace}
\vspace{\stretch{1}}\null

% ABSTRACT
%---------------------------------------------------------------------------

\cleardoublepage

\subfile{chapters/abstract.tex}

% }}}

% MAIN MATTER {{{
%---------------------------------------------------------------------------

\mainmatter

\subfile{chapters/introduction.tex}

\subfile{chapters/methods.tex}

\subfile{chapters/gamazon2015.tex}

\subfile{chapters/gusev2016.tex}

\subfile{chapters/gusev2018.tex}

\subfile{chapters/conclusion.tex}

% }}}

% BACK MATTER {{{
%---------------------------------------------------------------------------

\backmatter

% BIBLIOGRAPHY
%---------------------------------------------------------------------------

\printbibheading[heading=bibintoc,title=References]
\printbibliography[heading=subbibliography,keyword=TWAS,title=Main 
Articles]
\printbibliography[heading=subbibliography,notkeyword=TWAS,title=Further 
Reading]
\nocite{*}

% APPENDIX
%---------------------------------------------------------------------------

%\appendix

% INDEX
%---------------------------------------------------------------------------

%\printindex % Print the index at the very end of the document (remember 
%to uncomment \makeindex if you use the index)

\end{document}

% }}}
