\documentclass[../main.tex]{subfiles}
\begin{document}

\chapter{Introduction}
\labch{intro}

Genotype => Expression => Phenotype <= Environment

Actually, environment can influence gene expression and epigenome as 
well. Also the genome, for instance UV rays cause mutations. Most of the 
time the environmental effects are random, but not always (\eg UV rays, 
smoking...).

gamazon2015:Gwas have found many associations, but a large sample size 
is needed.

other limit of gwas: they study single variants, but sometimes the 
disease manifest only when there is a certain \textit{combination} of 
variants.

gamazon2015: Gwas on their own are not enough (cite 
https://www.nature.com/articles/nature08494). in particular, there is a 
missing link between the variant and the disease: how (not why, 
\textit{how}) does the variant make one individual more susceptible to a 
disease? It is not true that the nearest gene is always involved.

fine mapping? (it may be necessary also for TWAS.)

gamazon2015: Many SNPs are found in regulatory regions, as evinced by 
the fact that they overlap with DNaseI sites (is this true? read Gusev, 
A. et al. Regulatory variants explain much more heritability than coding 
variants across 11 common diseases. bioRxiv 004309 (21 April 2014).), 
and that they often are found in eQTL (see Nicolae, D.L. et al. 
Trait-associated SNPs are more likely to be eQTLs: annotation to enhance 
discovery from GWAS. PLoS Genet. 6, e1000888 (2010).)

gamazon2015: eQTL mapping has shown that intermediate phenotypes, 
especially gene expression, are important. New projects are generating 
huge amounts of epxression data: ENCODE, GTEx, GEUVADIS, to name a few.

gamazon2015: they provide a way to aggregate SNPs in a biologically 
meaningful way, since they combine those variants that influence the 
expression of a gene. In general, either the phenotype is influenced by 
a small number of variants (one in the extreme case of mendelian 
diseases) with a large effect, or by a large number of variants of small 
effect. PrediXscan should districate this entanglement.

marker1
marker2
			} => gene expression => phenotype
marker3
marker4

gamazon2015: multiple testing problems reduced.

\end{document}
