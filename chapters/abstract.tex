\documentclass[../main.tex]{subfiles}
\begin{document}

\chapter*{Abstract} % "\chapter*" leaves out this chapter from the toc

Understanding how genetic variation among individuals can influence the 
manifestation of complex diseases, which stem from the interaction of 
many genes with each other and with the environment, is a relevant 
problem in medicine. Since the first genome-wide association study (or 
GWAS) was conducted in 2005, tens of thousands of SNP-trait associations 
have been reported, shedding light on the at least partly genetic roots 
of many diseases; most of such associations, however, do not provide 
much predictive value and are difficult to explain. Expression 
quantitative trait loci (eQTL) mapping, which identifies loci that 
influence gene expression, is a possible step towards a better 
understanding of the relationship between genetic variation and 
phenotypic trait, using gene expression as a proxy for the trait. 
Recently, a new approach has been devised which goes one step further 
and aims to directly find associations between the expression of each 
gene and a given trait by combining GWAS and eQTL data. In one of their 
versions, these \enquote{transcriptome-wide association studies} (or 
TWAS) are performed in two phases, the first being the prediction of the 
genetic component of gene expression of the individuals in a GWAS cohort 
using reference transcriptome data, and the second being the evaluation 
of the association between predicted expression and trait in those 
individuals. On the whole, TWAS are powerful statistical methods to find 
associations between gene expression and complex phenotypic traits; 
while they can help in making sense of GWAS results, they also can find 
novel associations, pointing at potential candidate genes for further 
analysis: as such, their contribute to the characterisation of the 
relationship between genome and phenotype is substantial. After an 
introduction, the focus of the first part of this thesis will be a 
method to leverage individual-level data in order to detect genes 
associated with disease traits. The second part shall deal with how, 
conveniently, a TWAS can be performed starting only from the summary 
association statistics and the summary LD information of a GWAS. In the 
third part, we will discuss the advantages of integrating epigenetic 
markers in a TWA study and see an application to schizophrenia. 

\end{document}
