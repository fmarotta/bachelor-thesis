\documentclass[../main.tex]{subfiles}
\begin{document}

\chapter[A gene-based association method for mapping traits]{A 
	gene-based association method for mapping traits using reference 
	transcriptome data}
\labch{gamazon2015}

\subsection{Eric R. Gamazon \etal (2015), Nature Genetics}

\begin{external_abstract}{title=\textit{Abstract}}
Genome-wide association studies (GWAS) have identified thousands of 
variants robustly associated with complex traits. However, the 
biological mechanisms underlying these associations are, in general, not 
well understood. We propose a gene-based association method called 
PrediXcan that directly tests the molecular mechanisms through which 
genetic variation affects phenotype. The approach estimates the 
component of gene expression determined by an individual's genetic 
profile and correlates \enquote{imputed} gene expression with the 
phenotype under investigation to identify genes involved in the etiology 
of the phenotype. Genetically regulated gene expression is estimated 
using whole-genome tissue-dependent prediction models trained with 
reference transcriptome data sets. PrediXcan enjoys the benefits of 
gene-based approaches such as reduced multiple-testing burden and a 
principled approach to the design of follow-up experiments. Our results 
demonstrate that PrediXcan can detect known and new genes associated 
with disease traits and provide insights into the mechanism of these 
associations.
\end{external_abstract}

\section{Introduction}

% GWAS needs large sample size because the effects are small. sequencing 
% of large sample size is unfeasible.

Albeit it is accepted that in the majority of cases the biological role 
of variants associated to diseases is regulatory, as confirmed by the 
fact that many such variants fall in regions that are epigenetically 
marked as regulatory, GWAS results remain mainly uncharacterised from a 
functional point of view, and are only able to explain a little 
proportion of phenotypic variance. The wealth of biological data that is 
now being released by large-scale consortia provides an unprecedented 
opportunity to integrate information and obtain insight into the genetic 
and biological processes underlying disease 
susceptibility\sidenote[][-3cm]{Some of these consortia, whose data sets 
have been used by Gamazon \etal, are the following.
\begin{description}
	\item[ENCODE.] The focus is on the systematic functional annotation 
of each segment of the human genome.
	\item[GEUVADIS.] This project endeavours to uncover functional 
variation in humans through the study of how genetic variants affect 
gene expression.
	\item[DGN.] Variants regulating gene expression, splicing and 
allelic expression are detected.
	\item[Braineac.] The authors find eQTL in ten human brain regions.
	\item[GTEx Project.] Its aim is to collect data on genotype and gene 
expression levels of a number of tissues from postmortem samples.
\end{description}
}.

This work is based on two key ideas: first, genetic variants most often 
impact gene expression, as shown by the many eQTL studies; second, SNP 
aggregation methods that combine many variants in a biologically 
meaningful way have the potential to improve GWAS results. Therefore, 
the authors propose to group together all the SNPs that regulate the 
expression of a given gene. One advantage of this approach, which they 
called PrediXcan, is that statistical tests performed on group of SNPs 
are more powerful than those performed on each and every SNP due to less 
multiple testing; besides, by choosing the gene as a grouping unit, 
information about the directionality of the effect is intrinsically 
provided, \ie it is possible to say whether the disease is associated to 
an increase or a decrease in the gene's expression. Moreover, from a 
functional point of view a gene is much more interpretable than a simple 
genetic polymorphism.

\section{Imputation of gene expression}

The main goal is to provide a framework to better interpret GWAS 
results. However, in a typical GWA study the individuals are simply 
genotyped with a SNP microarray, and expression data are completely 
missing. Therefore, gene expression has to be predicted by exploiting 
the knowledge of expression quantitative trait loci.

The first assumption here is that gene expression can be decomposed into 
three components: genetically regulated expression (GReX), 
phenotype-influenced expression, and an environment-determined component 
(\reffig{gamazon2015/1}). Some phenotype, such as many diseases, can 
indeed influence gene expression, but, as we shall see in a moment, by 
training their predictive models on healthy individuals from reference 
transcriptome experiments, they already exclude that component from what 
they predict.

\begin{figure}
	\includegraphics[width=0.8\textwidth]{gamazon2015/1-expression_decomposition}
	\caption{Gene expression can be decomposed into three components}
	\labfig{gamazon2015/1}
\end{figure}

The prediction relies on data sets where both genotype and expression 
are present, such as the aforementioned GEUVADIS and GTEx projects, and 
the model is additive (\reffig{additive_model}), meaning that a given 
variant in a homozygous individual is supposed to have twice the effect 
of that same variant in an heterozygous individual. This is surely an 
oversimplification, for it does not take into account three biologically 
important effects ---epistasis, dominance and penetrance---, but an 
additive model is much simpler to implement. Moreover, the use of 
multiple linear regression, though a simplistic approach, is a first 
attempt to quantitatively model the interactions among genetic variants: 
indeed, it may well be that a given phenotype is influenced by a 
\textit{combination} of SNPs rather than a single SNP\todo{safety 
check.}. The purpose of this regression model is to find for each SNP 
the coefficient of which gene expression is altered by a copy of that 
SNP. Once the coefficients have been estimated, the genetically 
regulated component of gene expression can be predicted starting only 
from the genotype of an individual; the predicted GReX is denoted as 
$\widehat{GReX}$.

\begin{marginfigure}
	\includegraphics{gamazon2015/ext-additive_genetic_model}
	\caption{An example of additive model for one SNP; Gamazon \etal 
extended it for several SNPs. Image adapted from Conall M. O'Seaghdha 
and Caroline S. Fox, \enquote{Genome-wide association studies of chronic 
kidney disease: what have we learned?}}
	\labfig{additive_model}
\end{marginfigure}

They thus generated predictDB, which stores the coefficients of which 
each SNP influence the GReX. By using healthy individuals from reference 
transcriptome and genome data sets, they disregard the 
disease-determined component of gene expression, and by using a 
regression model, they disregard the random environmental component. It 
is now possible to \enquote{impute} the transcriptome of an individual 
from its genotype, just like it has been possible to impute unknown 
variants in an individual from its known genotyped variants.

\begin{figure}
	\includegraphics[height=10cm]{gamazon2015/2-grex_estimation}
	\caption{The framework to estimate the coefficient by which each SNP 
alters the expression of a gene.}
	\labfig{gamazon2015/2}
\end{figure}

The regression model employed can be summarised with the following 
equation (referring to \reffig{gamazon2015/2}):

\begin{equation}
	T = \sum_{k=1}^{M}w_k X_k + \epsilon
\end{equation}

where T is the expression of a given gene in a given tissue, $w_k$ is 
the weigth (or the effect size) of SNP $k$ in influencing the expression 
of that gene, $X_k$ is the number of reference alleles of SNP $k$ in all 
the dataset, and $\epsilon$ is a random error. Only SNPs falling 1Mb 
within the gene's start or end were considered. To fit this model, the 
authors tried various types of regression methods ---lasso, elastic net 
and polygenic scrores---, but in the end settled to elastic 
net\sidenote{In general, elastic net is used for two reasons: first, 
when the number of predictors is large, especially if compared to the 
number of samples; and second, to avoid overfitting.} (with the 
parameter $\alpha = 0.5$), whose main advantage is the 
\enquote{automatic} selection of the most important regressors. 10-fold 
cross-validation\sidenote{In $k$-fold cross-validation, the dataset is 
split in $k$ portions, and for each part, the model is trained on the 
remaining $k-1$ parts, then the $R^2$ of the predicted and real values 
is calculated on the selected part. The average of the $R^2$ is finally 
reported.} was used to asses the predictive performance.

Having seen the general method, we now turn to the actual protocol 
followed by the authors.

From DGN, they obtained whole-blood RNA-seq and genome-wide genotype 
data for 922 European individuals, normalised and filtered to retain 
only SNPs with MAF > 0.05, in Hardy-Weinberg equilibrium and univoquely 
mapped onto a strand. The SNPs that were not genotyped were imputed; 
genome imputation is a routinely-used method to estimate the genotype of 
an individual at loci that were not analysed, basing on known linkage 
disequilibrium information in a reference population. This data set was 
used to train the predictive models.

From GEUVADIS, normalised RNA-seq data from lymphoblastoid cell lines 
established from 421 individuals was downloaded; genotype data was also 
available for the same individuals, since they were part of the 1000 
genomes project as well. This data set was used for the validation of 
the models (\reffig{gamazon2015/4}).

In the quantile-quantile plot each point represents an expression value; 
on the $x$-axis it is reported the $R^2$ that would be expected if the 
null hypothesis were true (\ie, if there were no correlation between 
predicted and observed expression values), while on the $y$-axis there 
is the observed $R^2$. A 45\textdegree line is displayed in gray: the 
farther the points are from that line, the more different the two 
distributions of $R^2$ are.\todo{safety check}

RNA-seq data from GTeX, normalised and adjusted for the most common 
covariates such as sex, was used to test the predictive performance of 
the model in different tissues. Surprisingly, the model was able to 
predict gene expression quite accurately in tissues other than the 
blood, even though it was trained on blood expression data.

\begin{figure}
	\includegraphics{gamazon2015/4-prediction_performance}
	\caption{The performance in the GEUVADIS data set of the elastic net 
model trained on the DGN data set, showed by a quantile-quantile plot 
(left) and a distribution of $R^2$ (right).}
	\labfig{gamazon2015/4}
\end{figure}

\section{Heritability of gene expression}
\labsec{heritability}

Heritability is an important idea in genetics and is especially relevant 
in the scope of association studies, therefore we dedicate some space to 
its analysis.

Many traits vary among the individuals of a population: height and hair 
colour are obvious ones, but for instance also disease status (or 
liability) can be considered a phenotypic trait. The heritability of a 
trait is the proportion of trait variance which can be explained by the 
genetic variance among the individuals of the population. In order not 
to underestimate heritability, only genetic variance at loci associated 
to the phenotype must be taken into account\todo{safety check}. 
Heritability does not deal with the influence of genes in the 
development of the trait, but rather is concerned with the role of 
genetic \textit{variation} in determining phenotypic variation. There 
are two definitions of heritability:

\begin{description}
	\item[Narrow sense heritability,] or $h^2$, is the heritability due 
to additive genetic factors (see \reffig{additive_model} and related 
discussion).
	\item[Broad sense heritability,] or $H^2$, is the heritability due 
to all genetic factors, taking into account dominance and gene-gene 
interactions.
\end{description}

\marginnote[-1.5cm]{If we assume that $P = G + E$, where $P$ is the 
	phenotype, $G$ the genetics, $E$ the environment, then phenotypic 
	variance can be expressed as follows:
\begin{equation*}
	Var(P) = Var(G) + Var(E) + 2 Cov(G,E)
\end{equation*}
and, assuming the independence of genetics and environment,
\begin{flalign*}
	&H^2 = Var(G) / Var(P)
	&\\
	&h^2 = Var(A) / Var(P)
\end{flalign*}
}

Usually, the first definition is used, but it is a reasonable 
approximation, for in the majority of case, alternate alleles are 
homozygous only in a minority of individuals due to their low frequency, 
hence the effects of dominance or epistasis manifest only 
rarely\cite{Visscher2008}\todo{safety check}.

There are many ways to estimate narrow-sense heritability. One is in 
selective breeding, where the heritability is the proportionality 
coefficient between the intensity of the applied selective pressure, 
$S$, and the response to selection, $R$ (\reffig{breeders}). In other 
words, $R = h^2 S$, wihch is the famous breeder's equation. The larger 
the heritability, the greater the response to selection.

\begin{figure}
	\includegraphics[width=0.8\textwidth]{gamazon2015/ext-breeders}
	\caption{The breeder's work. Source: 
\url{https://wiki.groenkennisnet.nl/}}
	\labfig{breeders}
\end{figure}

Another way to interpret heritability in the narrow sense is the 
plotting of the averaged trait in the two parents versus the trait in 
their offspring. In principle, if genes determine variations in 
phenotypes, then offspring should be more similar to their parents than 
to unrelated individuals. This can be expressed as a correlation between 
parents and offspring (\reffig{parents_offspring}). In other words, if 
we have $X = "phenotype of parent X"$ and $Y(X) = "phenotype of 
offspring of parent X"$, then the plot of $Y(X)$ should be a straight 
line with positive slope, and in the extreme case where there is perfect 
heritability, the plot should be of the form $Y(X) = X$. In general, 
$h^2$ is the slope of the regression line. This, however, is valid only 
if the environment is not shared between relatives more than it is 
shared between unrelated people.

\begin{figure}
	\includegraphics{gamazon2015/ext-parents_offspring}
	\caption{Parents-offspring regression. Source: Visscher \etal 2008, 
\enquote{Heritability in the genomics era — concepts and 
misconceptions}}
	\labfig{parents_offspring}
\end{figure}

Returning to the paper by Gamazon \etal, they rightly claim that 
heritability is an upper bound to how well the trait can be associated 
to the genotype. A high heritability means that the parents' trait can 
predict the offspring trait, or, equivalently, since this predictability 
is due to genetic factors, that people with a similar genotype will have 
a similar phenotype (indeed, $h^2$ is precisely the correlation between 
the phenotypes in parents and offspring).

The heritability of gene expression in DGN cells was computed, resulting 
in an average value of 0.153, whereas the average 10-fold 
cross-validation $R^2$ with elastic net was 0.137. 
\reffig{gamazon2015/3} reports the results of the cross-validation.

\begin{figure}
	\includegraphics{gamazon2015/3-prediction_r2}
	\caption{$R^2$ (red) of $\widehat{GReX}$ versus observed expression; 
heritability (black) of gene expression.}
	\labfig{gamazon2015/3}
\end{figure}

\section{Correlation between expression and phenotype}

In the second phase, the predicted $\widehat{GReX}$ is correlated with 
the phenotypic status employing linear regression, logistic regression, 
Cox regression, or Spearman correlation (the latter is non-parametric). 
They chose to use logistic regression for the results discussed in this 
article. Another possibility could have been that to model not the 
disease status, which is a binary variable, but rather the liability to 
the disease, following what Vissher\cite{Visscher2008} reported, showing 
that this approach is able to explain a larger proportion of genetic 
variance.

\section{Application of PrediXcan to WTCCC}

At last, their method was applied to seven autoimmune diseases which had 
previously been the object of as many GWAS by the Wellcome Trust Case 
Control Consortium. They used DGN whole-blod elastic net prediction 
models to predict the expression in each WTCCC cohort, then correlated 
the predicted GReX with the disease status. After applying Bonferroni 
correction, 41 significant associations (P < 0.05) with 5 diseases were 
reported. Most of the significant associations were for autoimmune 
disease and were located in the extended MHC 
region\sidenote[][-0.5cm]{This region, located on chromosome 6, harbours 
	421 loci, including 252 expressed genes, 139 pseudogenes and 30 
	transcripts. Many of these loci are associated to diseases.} 
Moreover, some genes were associated to multiple diseases\sidenote{In 
	these cases, what determines which disease shows up if the 
	expression of that gene is altered in an individual? Perhaps the 
	environment, or gene expression level. This is an example of the 
	complexity of the situation: the relationship between genotype and 
	phenotype is not biunivocal at all.} The majority of these 
associations were supported by previous evidence, and often they were 
enriched in known GWAS; but two completely novel disease-associated 
genes were also found: low expression of \textit{KCNN4} was associated 
with hypertension, and high expression of \textit{PTPRE} with bipolar 
disorder.

One of the most interesting features of this new approach is that not 
only can it provide association results, but also a directionality. As 
an example, \textit{PTPN22} expression was positively associated with 
rheumatoid arthritis and type 1 diabetes, and negatively associated with 
Crohn's disease. The results of the associations with type 1 diabetes 
are reported in \reffig{gamazon2015/7}.

\begin{figure}
	\includegraphics{gamazon2015/7-t1d_associations}
	\caption{(a) Manhattan plot of disease-gene associations P-values. 
(b) Q-Q plot of the same P-values and top three genes.}
	\labfig{gamazon2015/7}
\end{figure}

\section{Discussion}

In this paper, the authors developed a framework to group together 
genetic variants, predict how such variants influence gene expression, 
and finally correlate gene expression to a disease. This approach has 
many advantages. First of all, it provides the directionality of the 
effect of a gene on the disease, hinting at potential strategies to cure 
the disease: for instance, if a gene's over-expression is linked to a 
disease, then a drug may be developed to downregulate it. Furthermore, 
directionality can also give insight into the underlying causal 
mechanism of the disease.

An added benefit of grouping many variants together is that there will 
be less multiple testing correction to perform, leading to more variants 
that exceed the significancy threshold. A further step could be to group 
together many genes into the same pathway.

Finally, an important part is that an analysis with PrediXcan is 
economic, in the sense that one only needs reference transcriptome and 
GWAS data, which are already available; therefore, many existing GWAS 
dataset can be reanalysed \enquote{for free}.

However, there are also some limits. For instance, the prediction of 
gene expression can be biased; in this paper elastic net was employed, 
but some models, namely a combination of K nearest neighbour (KNN), 
elastic net, and the use of genomic annotation may perform better. 
Besides, elastic net does not take into account biologically relevant 
concepts such as epistasis, dominance and penetrance.

%(https://www.cell.com/ajhg/fulltext/S0002-9297(18)30108-3?code=cell-site) 
Another possible critic to this work is that genetic variation does not 
alter only gene expression. There can be \textit{trans}-acting effects, 
where a SNP alters how a gene (be it a TF or a miRNA) modulates the 
expression of others, without altering the expression of the modulator 
gene itself. Moreover, a SNP can have effects on splicing, transcription 
start or end site or other RNA editing processes, without altering the 
expression of the gene.

Even with this method, then, causal SNP or mechanisms cannot be 
inferred. Indeed, disease-associated SNPs may contribute both to gene 
expression and to other factors that impact the disease independently 
from gene expression. PrediXcan  only says that variation between 
individuals at that locus results in variation in gene 
expression.\todo{safety check}

\end{document}
