\documentclass[../main.tex]{subfiles}
\begin{document}

\chapter{Seminal work: TWAS from individual-level data}
\labch{gamazon2015}

\subsection{Abstract}

\enquote{\footnotesize Genome-wide association studies (GWAS) have 
	identified thousands of variants robustly associated with complex 
	traits. However, the biological mechanisms underlying these 
	associations are, in general, not well understood. We propose a 
	gene-based association method called PrediXcan that directly tests 
	the molecular mechanisms through which genetic variation affects 
	phenotype. The approach estimates the component of gene expression 
	determined by an individual's genetic profile and correlates 
	'imputed' gene expression with the phenotype under investigation to 
	identify genes involved in the etiology of the phenotype. 
	Genetically regulated gene expression is estimated using 
	whole-genome tissue-dependent prediction models trained with 
	reference transcriptome data sets. PrediXcan enjoys the benefits of 
	gene-based approaches such as reduced multiple-testing burden and a 
	principled approach to the design of follow-up experiments. Our 
	results demonstrate that PrediXcan can detect known and new genes 
	associated with disease traits and provide insights into the 
	mechanism of these associations.}

\section{Intro}



\end{document}
