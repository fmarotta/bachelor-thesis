\documentclass[../main.tex]{subfiles}
\begin{document}

\chapter{Conclusions and future perspectives}
\labch{conclusion}

Transcriptome-wide association studies were born to address some of the 
limitations of classical association studies by combining many SNPs in a 
biologically meaningful way, \ie through their effect on the expression 
of a gene. However, one first problem was the paucity of large-cohort 
studies where genomic, transcriptomic and phenotypic data are collected 
simultaneously; on the other hand, GWAS, where only genotype and 
phenotype are characterised, are very common. This problem was solved by 
imputing gene expression with a regression model able to predict the 
levels of expression starting from genotypic information. But another 
problem arised: in published GWA studies the full genotype of each 
individual is not available; in order to solve this problem, the 
association between gene expression and trait was performed at the level 
of the whole GWAS cohort, still relying on reference transcriptome data 
sets to grasp the relationship between SNPs and expression.

The aggregation of SNPs in functional units ensure that TWAS perform 
better than GWAS when there are multiple causal SNPs (see 
\refsec{allelic_heterogeneity}). Statistically, TWAS are more powerful 
than single-variant-based approaches, for the multiple testing buden is 
reduced. Moreover, the interpretability of results increases and 
directionality information is provided. From a human point of view, 
knowing that having a \enquote{G} rather than a \enquote{T} at a locus 
makes one more liable to a disease does not make much sense. On the 
contrary, knowing that one is at risk if a gene is more (or less) 
expressed than normal, is somewhat reassuring.

The other side of the coin is that TWAS miserably fail when genetic 
variants influence the phenotype independently of gene expression. Even 
worse, if a variant pleiotropically affects both gene expression and 
disease independently, TWAS are confounded, in the sense that they 
report an association between expression and disease, where in fact 
there is none.

A genetic variant can affect phenotypes in many ways, which fall in two 
main broad categories: a regulatory or a structural one. In the paper by 
Gusev \etal on schizophrenia, the possibility of structural alterations 
was explored by associating the expression of particular spliced 
isoforms to the disease, to perform a \enquote{spliceome-wide 
	association study}. In the same paper, gene expression was 
associated to chromatin activity. These examples highlight the fact that 
association studies can go beyond genetic variants alone. In principle, 
every intermediate phenotype in the path from genome to disease could be 
associated with the disease status. A possible future perspective is to 
integrate associations from more than one phenotypic levels (see 
\refsec{phenolevels}). Indeed, genetic variants can have different 
effects in different contexts.

An advantage of performing associations using gene expression is that 
this intermediate phenotype is fairly heritable, and above all its 
prediction from genetic variants explains a good proportion of such 
heritability. On the contrary, higher-level phenotypes are difficult to 
predict from the genotype. This could mean that there is a more direct 
link between gene sequence and gene expression than between gene 
sequence and height. The \enquote{missing heritability}, not explained 
by GWAS, could be lost in processes that do not depend directly on gene 
sequence: each time a phenotypic level is crossed, the possible 
combinations expand.

Is there a way to understand the mechanisms behind association and to 
find causal variants? This should remain an open question for now. 
Although transcriptome-wide associations can lead to putative causal 
genes, they are cannot replace experimental validation. Fine mapping of 
causal genes is still necessary.

But the fact that we do not know the mechanism does not prevent us from 
exploiting existing associations. For instance, if the expression of a 
gene is positively correlated with the risk of a disease, we could 
develop drugs that down-regulate that gene. Not only are genes more 
interpretable: they are also more druggable than SNPs.

In principle, any population whose genotypic and phenotypic data are 
available can be used to perform a TWAS. For instance, one could start 
from a population of tumours, such as those collected by TCGA, and 
leverage this data to find driver genes. Another possibility is to start 
from a population of single cells and ask which are the genes that make 
each of them unique.

A possible limitation of TWAS is as follows. If a disease-causing 
variant alters the activity of a transcription factor, and this altered 
transcription factor affects the expression of hundreds of genes, then 
every one of these genes would be associated to the disease, albeit the 
\enquote{true} causal gene was that encoding for the transcription 
factor. Moreover, multiple variants in different individuals can lead to 
the same disease, because they alter either the same gene or different 
genes partaking in the same functional pathway or coregulated. All this 
heterogeneity decreases the statistical power to find associations. 
Perhaps one of the possible solutions is to aggregate genes in even 
higher level biological units, like functional pathways. Such 
\enquote{network approach} seems very promising, for it naturally 
accounts for the fact that the behavious of living systems emerge from 
the interactions of its elements.

On the whole, transcriptome-wide association studies are one of the 
methods through which the relationship between genome, transcriptome and 
disease can be investigated. They are already applied quite widely and 
hopefully in the future will enable us to understand and cure many 
diseases.

\end{document}
