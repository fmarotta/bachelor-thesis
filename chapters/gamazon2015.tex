\documentclass[../main.tex]{subfiles}
\begin{document}

\chapter{Seminal work: TWAS from individual-level data}
\labch{gamazon2015}

\subsection{Abstract}

\enquote{\footnotesize Genome-wide association studies (GWAS) have 
	identified thousands of variants robustly associated with complex 
	traits. However, the biological mechanisms underlying these 
	associations are, in general, not well understood. We propose a 
	gene-based association method called PrediXcan that directly tests 
	the molecular mechanisms through which genetic variation affects 
	phenotype. The approach estimates the component of gene expression 
	determined by an individual's genetic profile and correlates 
	'imputed' gene expression with the phenotype under investigation to 
	identify genes involved in the etiology of the phenotype. 
	Genetically regulated gene expression is estimated using 
	whole-genome tissue-dependent prediction models trained with 
	reference transcriptome data sets. PrediXcan enjoys the benefits of 
	gene-based approaches such as reduced multiple-testing burden and a 
	principled approach to the design of follow-up experiments. Our 
	results demonstrate that PrediXcan can detect known and new genes 
	associated with disease traits and provide insights into the 
	mechanism of these associations.}

\section{Method}

first step: gene expression is decomposable in three components: genetically
regulated expression (GReX), phenotype-influenced expression, and an
environmental component. The phenotype can influence gene expression.

An additive model trained on reference transcriptome datasets finds for each
SNP the coefficents of which gene expression is altered by each SNP, i.e. it
says that SNP rs483920482905, when present in an individual, alters the
expression of gene XXXX by a factor 1.5. Clearly, the training dataset must
contain both genome and transcriptome data. Afterwards, the GReX is predicted
in indivuduals for which only the genome sequence is available. They thus
generated predictDB.

\begin{equation}
	T = \sum_{k=1}^{M}{w_k X_k + \epsilon}
\end{equation}

T is the expression of a gene, $w_k$ is the weigth of SNP k in influencing the
expression of that gene, and $X_k$ is the number of reference alleles of SNP k
(I guess $X_k$ is the sum of the alleles in all the individuals in the
dataset).

consideration by fmarotta: there are other models available, for instance one
could account for the penetrance, or use a dominant (recessive) model, and so
on.

Important (by fmarotta): in linear regression, each regressor is considered
independent of the others, but is that so? I think often a phenotype can depend
on the \textit{combination} of SNPs.

(also by fmarotta): is it possible to use PCA to find which SNPs are most
relevant in influencing a phenotype?

In the second phase, the predicted GReX is correlated (with linear regression,
logistic regression, Cox, or Spearman (the latter is non-parametric). They used
logistic regression for the results discussed in this article.

limits: there is an attenuation bias because of the error in the estimation of
the GReX.

\section{features}

The rationale for everything is that often SNPs influence a phenotype by
altering gene expression (i.e. they have regulatory roles), as stated in this
article:
http://journals.plos.org/plosgenetics/article?id=10.1371/journal.pgen.1000888

main advantages:

\begin{itemize}
\item directionality, possibility to get insights on mechanisms.
\item small multiple-testing burden.
\item possibility form functional units (e.g. basing on known pathways).
\item possibility to reanalise gwas data (only the genome is needed).
\end{itemize}

\section{Prereq: 
https://www.sciencedirect.com/science/article/pii/S0002929710005987?via%3Dihub}

\subsection{pre-prereq: https://www.nature.com/articles/nature08494}



\section{Predicting the transcriptome}

TODO: understand LASSO and elastic net!

They chose to use elastic net and used tenfold cross-validation (\ie 
they looked at the R square of estimated GReX vs observed expression).

They also computed the heritability of gene expression in DGN and claim 
that 

\end{document}
