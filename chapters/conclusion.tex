\documentclass[../main.tex]{subfiles}
\begin{document}

\chapter{Conclusions and future perspectives}
\labch{conclusion}

Transcriptome-wide association studies were born to address some of the 
limits of classical association studies by combining many SNPs in a 
biologically meaningful way, \ie through their effect on the expression 
of a gene. However, one first problem is the paucity of large-cohort 
studies where genomic, transcriptomic and phenotypic data are present 
simultaneously (on the other hand, GWAS, where only genotype and 
phenotype are characterised, are very common). This problem was solved 
by imputing gene expression with a regression model able to predict the 
levels of expression starting from the genotypic information. Bun 
another problem arises: in published GWA studies the full genotype of 
each individual is not available; in order to solve this problem, the 
imputation of gene expression is performed at the level of the whole 
GWAS cohort.

A variant can have two main roles: a regulatory one and/or a structural 
one. If a variant affects a TF, it can indirectly be associated with 
many genes, and potentially many diseases. Different variants can affect 
the same genes, so the same disease can be due to many variants, thus 
increasing the difficulty in finding associations: think about how many 
variants can in principle affect the expression of a gene! Regulatory 
role: it alters a binding site. Structural role: ??

limits: assumption of linearity.

Who regulates who? Network approach: prioritisation of interactions, 
prioritisation of combinatoric effects.

%One could find genes near markers associated to a phenotype and look 
%for overrepresented pathways. Nobody did it here.

\end{document}
