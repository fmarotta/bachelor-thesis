\documentclass[../main.tex]{subfiles}
\begin{document}

\chapter{Linkage Disequilibrium}
\labch{ld}

% Definition {{{

\section{Definition}

Linkage disequilibrium is the non-random association of two alleles. In
other words, two alleles can be found together in more (or less)
individuals than what is expected by chance.

An important concept is that linkage disequilibrium always refers to
alleles in the same haplotype, i.e. in the alleles which are inherited
together from a single parent. For haployd organisms, therefore,
we will consider the genome of the individual as haplotype, but for
diploid individuals, such as humans, we can consider the gametes.

Essentially, LD is maintained because haplotypes are inherited together.

As is often the case, a great deal of information can be extracted from 
systems that are not at equilibrium.

% }}}

% Wiley article {{{

\section{http://www.els.net/WileyCDA/ElsArticle/refId-a0005427.html}
\labsec{ld_balance}

Though most of the alleles in linkage disequilibrium are found on the 
same chromosome, this is not always the case, for linkage disequilibrium 
can arise from many different processes: genetic mutations, selection, 
genetic drift and gene flow, to name a few. \textit{De novo} genetic 
mutation, where a new allele is created is an important one in linkage 
disequilibrium. It is clear that, among the descendants of the 
individual where the mutation has appeared, this allele is in linkage 
disequilibrium with its neighbours, for it always occurs with the same 
alleles.

Now let us consider selection; it is not difficult to imagine that the 
function of a protein complex made by two subunits is affected by which 
allele is present for each of the two genes involved in the complex, 
therefore it is conceivable that those combinations that work best will 
be selected. Linkage disequilibrium can be experimented by non-coding 
regions as well, but such situation is more complex to depict.

Another interesting process which influence the equilibrium between loci 
is recombination, which acts during meiosis. Recombination is different 
from the previous processes, for it does not create disequilibrium, but 
rather dissipates it, bringing the alleles back to equilibrium with a 
rate which is a function of the phyiscal distance between the loci. It 
follows that if we know the recombination rate between two loci and 
their linkage disequilibrium, we can estimate the time at which the 
alleles appeared. Another consequence of recombination is the creation 
of recombinant phenotypes, which occur indeed when disequilibrium is 
broken. Let us suppose that allele \geno{A} is in perfect linkage 
disequilibrium with allele \geno{B}, so that they are always found 
together in the same haplotype; the same holds for \geno{a} and 
\geno{b}. When two different haplotypes, one having \geno{AB} and the 
other \geno{ab}, are found in the same diploid individual, a 
recombination event can occur between these two sister chromatides such 
as to cut the DNA right between locus A and locus B. In this case, a new 
phenotype, deemed "recombinant", would be created.

To sum up, disequilibrium between alleles is the product of many 
balancing processes; recombination, on the other hand, balances those 
processes so as to bring equilibrium. A corollary of these observations 
is that if two alleles are in disequilibrium, in all probability there 
is a reason behind, be it that one of the allele is newly appeared or 
that there is selection to maintain the two alleles together.

Linkage can be calculated for more than two alleles, being on the same 
locus or in different loci, but it is rarely done. I do not know whether 
it is due to the complexity of the calculation or because it has a poor 
biological relevance, but I would bet that it would be an interesting 
area to explore.

% }}}

% PIFFLE {{{

\section{http://www.handsongenetics.com/PIFFLE/LinkageDisequilibrium.pdf}

% Measures of LD {{{

\subsection{Measures of Linkage Disequilibrium}

Let us consider a simplified model where all loci are independent, 
taking two loci, A and B, both of which are biallelic, carrying alleles 
\geno{A,a} and \geno{B,b}, respectively; also, in our model the 
population is a set of diploid individuals, each of which can be 
described by a pair of haplotypes. In all the gametes of such 
population, allele \geno{A} has a frequency of \geno{p_A}, and allele 
\geno{B} of \geno{p_B}, so it follows from simple probabilistic 
considerations that in a diploid organism the probability of finding 
both alleles is \geno{p_{AB} = p_a p_b}. This result, however, holds 
true only as long as the loci are independent.

In a real population, \geno{p_{AB}} may well differ from the product of 
the allele frequencies, so it has been introduced the `linkage 
coefficient', which is defined as

\begin{equation}
\label{eq:linkagecoeff}
D = p_{AB} - p_A p_B\,,
\end{equation}

as a measure of this difference.

Much as the linkage coefficient is useful as a measure of linkage 
disequilibrium, it has the disadvantage of being dependent upon allele 
frequencies. Indeed, its range can only be between \(p_A p_B\) and \(p_A 
	- p_B\), which is the reason why Lewontin\cite{lewontin1964} 
introduced D', defined as

\begin{equation}
\label{eq:linkagecoefflewontin}
D' = D / D_{max}\ ,
\end{equation}

where \(D_{max}\) is the theoretical maximum value that \(D\) can assume 
given the allele frequencies of the population. As \(-1 <= D' <= 1\), 
this is a completely normalised parameter; however, this parameter only 
considers what is the difference between the actual frequency of the 
\geno{AB} haplotype and the expected one, disregarding the unequality of 
the allele frequencies at that locus.

Another measure of linkage disequilibrium, which retains the information 
of how much the allele frequencies differ at that locus, is the 
correlation coefficient
\marginnote[-1cm]{The linear correlation coefficient \(r\), often called 
	Pearson's correlation coefficient, measures how strength is the 
	linear relationship between two variables, as well as determining 
	the direction of such relationship. The more two variables can be 
	plotted on a straight line, the greater the \(|r|\). In this case, 
	the two variables are the allele frequencies.
}
between the two alleles, \(r\), indroduced by Hill and 
Robertson\cite{hill1968}; most often, the correlation coefficient is 
used in its squared version:

\begin{equation}
\label{eq:squaredcorrelationcoeff}
r^2 = \frac{D^2}{p_A (1 - p_A) p_B (1 - p_B)}\ .
\end{equation}

One of the strength of this parameter is that it has been shown to 
depend almost entirely on \(N\) (population size), \(c\) and generation 
number. Since it does not completely buffer the effect of different 
allele frequencies, its range will be dependent upon \geno{p_A} and 
\geno{p_B}.

% }}}

% Modeling LD {{{

\subsection{Modeling D}

Let us try to explain how \(D\) can be different from \(0\) by 
considering genetic mutations, which are one of the processes through 
which linkage disequilibrium can arise, as we said in 
\refsec{ld_balance}. At the beginning, loci A and B are monomorphic, 
that is they both have only one allele. You can imagine each allele as 
being a sequence of letters, or a word, in which case each locus only 
has one word; for the sake of our metaphor, let us say that locus A 
holds allele (word) ELEPPANT while locus B holds allele (word) CANGAROO. 
Each and every individual has two haplotypes, and each haplotype holds 
the word ELEPHANT and CANGAROO. At one point, one individual exposes 
himself to the open sun, and an UV ray causes a \textit{mutation} in his 
DNA, so that now, in one of his haplotypes, locus A reads ELEPHANT, 
whereas in the other haplotype it is still ELEPPANT. Since the mutation 
happened in his gametes, his offspring inherits it. It turns out that 
having this mutation makes the individuals fitter, so its frequency 
increases overtime, until it becomes \geno{p_A} (remember: allele 
frequency is the number of gametes having that allele devided by the 
total number of gametes in the population). The frequency of the 
ELEPPANT allele, instead, is \geno{p_a}. At another point, another 
mutation happens in a haplotype with allele \geno{a}, creating the 
allele CAGGAROO at locus B. Let us call allele CANGAROO \geno{B}, with 
frequency \geno{p_B}, and the other \geno{b}. Now, then, there are two 
alleles for each locus, and three possible haplotypes: \geno{AB}, 
\geno{ab}, and \geno{aB}, the latter of which is the original one. These 
haplotypes can originate six possible genotypes, namely the combination 
with repetition of the three alleles for the two haplotypes of each 
individual. Let us consider haplotype \geno{ab}. In the generation in 
which the mutation CAGGAROO appears, there are 5 individuals, \ie 10 
gametes, and the allele frequencies are showed in the table at the 
margin.

\begin{margintable}
\begin{tabular}{|c|c|}
	\hline
	Allele	& frequency	\\
	\hline
	A		& 0.7		\\
	a		& 0.3		\\
	B		& 0.9		\\
	b		& 0.1		\\
	\hline
\end{tabular}
\caption{Sample alleles and their frequency}
\end{margintable}

There is only one haplotype \geno{ab}, for that mutation has just 
appeared, so that \geno{p_ab = 0.1}; more generally, for \textit{de 
	novo} mutations, the frequency of the genotype having that mutation 
is equal to the allele frequency of the mutation. However, \geno{p_a p_b 
	= 0.3 \cdot 0.1 = 0.03}, which is much smaller than \geno{p_ab}, 
therefore, the linkage coefficient is \(D = 0.1 - 0.03 = 0.07\). 
Selection and other processes will act to increase or decrease such 
disequilibrium.

Now let us consider the other process, recombination, which as we said 
always decreases disequilibrium. The first step of the argument is as 
follows. Take haplotype \geno{AB}, where alleles \geno{A} and \geno{B} 
are together: if recombination between A and B occurs with rate \(c\), 
then afterwards there will only be \(1-c\) \geno{AB} haplotypes. On the 
other hand, recombination is an exchange process, and here is where the 
second step of the argument comes: this time, let us consider a complete 
genotype, \ie two haplotypes. Every time there is an allele \geno{A} and 
an allele \geno{B}, if there is recombination, a new \geno{AB} haplotype 
will be formed. Therefore, in the next generation, we have

\[p'_AB = (1 - c) p_{AB} + c p_A p_B\].

Note that if an \geno{AB} haplotype is broken, but the other haplotype 
undergoing recombination contained allele \geno{A} or \geno{B}, the two 
events `cancel out' perfectly.

In the absence of mutations and assuming a population infinite in size, 
with no genetic fluxes and no selection, the allele frequencies do not 
change, so that \(p'_A = p_A\) and \(p'_B = p_B\), leaving the following 
result for \(D'\):

\[D_1 = (1 - c) D_0\ ,\]

which cllearly results in a geometric series, where the linkage 
coefficient at generation \(n\) is given by

\begin{equation}
\label{eq:linkagecoeffrecomb}
D_n = (1 - c)^n D_0\ .
\end{equation}

\(c\) being a small positive number, this equation implies that over 
generations the linkage coefficient will decrease until it reaches zero 
for \(n \to \infty\); the rate of decrease will be greater the higher 
\(c\) is. Thus, our mathematical model confirms what we said at the 
beginning.

% }}}

% Applications of LD {{{

\subsection{Applications of LD}

\todo{Shall we investigate further on this matter?}

Linkage disequilibrium is such an important concept because it is the 
bedrock of a number of applications. Interestingly enough, by knowing 
the recombination rate at two loci and the \(r^2\) for the alleles at 
those loci at a point in time, one can estimate the population size.

Another use of disequilibrium is that of estimating the time at which 
one population splitted in two; this is accomplished by comparing the LD 
measures at the same two loci in the two populations. If LD between the 
same alleles would be the same for the two populations, they would 
actually be one population\ldots

Finally, linkage disequilibrium is also the basis for association 
studies. These are statistical analysis whose purpose is to find out 
alleles occurring more often than expected by chance in individuals 
bearing a certain phenotype, implying that such alleles are somewhat 
correlated with the phenotype. Traditionally, the phenotypes 
investigated in association studies are pathological ones, both because 
they are more interesting and because they are easier to study, provided 
that they are not complex diseases. In later sections we will be 
focusing on this application.

% }}}

% }}}

\end{document}
